%%%%%
%%
%% This is intended as a nearly complete rules and scenario document
%% that you, the GM, change and complete for your game.  Various
%% comments suggest what parts can be removed or changed based on
%% common game variants.  For example, you will not need the rule for
%% player rooms if your gamespace is not open or otherwise doesn't
%% include living spaces.  Some possibly useful sentences/paragraphs
%% are simply commented out.
%%
%% Feel free to ignore these comments and just write what you want.
%%
%%
%%
%% These rules are merely an example ruleset.  Nothing in them is
%% sacred or even established by consensus.  They do not prescribe a
%% standard.  As a GM, you can change, remove, or rewrite whatever is
%% necessary to fit your game design.
%%
%% However, much of the material presented here, especially in the
%% Getting Started and Items Etc. sections, may be taken for granted
%% by many players.  This has two major implications:
%%
%% 1) Because of gradual shifts, longstanding biases, and first
%% impressions, different players (and GMs!) may have very different
%% assumptions about some details.  Do not brush off the sections that
%% "everyone knows."  Make sure everyone pays attention to the rules
%% as a whole.
%%
%% 2) If you have a clever idea to change some detail in the
%% fundamental parts of the rules, make sure to draw attention to it.
%% For example, if you modify the item bulkiness rules, don't just
%% gloss over your changes in the middle of a paragraph.
%%
%% You may want a section at the end as a reference to any fundamental
%% changes.  See "New Rule Summary" at the end.
%%
%%
%%
%% The martial combat system (and related health states and ranged
%% combat system) in this doc is known as "darkwater," named after The
%% Pirates of Darkwater, for which the first version was written.  It
%% is included as an example combat system.
%%
%%
%%
%% Basic guidelines for rules-writing:
%%
%% Use simple, concise, and precise language.  Avoid colloquialisms
%% and speech mannerisms.  Write in the second person.  Use the
%% imperative voice when possible.  Write like you are writing
%% directions.
%%
%% When you change contexts, like going from writing to the attacker
%% to writing to the defender, at least change paragraphs.
%%
%% When first introducing a major term and/or abbreviation, use bold
%% text.  Try to define terms before you use them.  Combined with
%% section/subsection/paragraph headings, this will make the rules
%% easier to skim for reference.  First establish context, then go
%% into detail.
%%
%% Be thorough, but do not ramble.  Try to make the spirit of the
%% rules clear in addition to the letter.  If the spirit is hard to
%% relay, then the rule may be too complicated.
%%
%%%%%

\documentclass[sheet]{grimrock}

%% document-wide tweaks
\interlinepenalty10000
\setstretch{1}
\def\mytype{Rules and Scenario}
\lfoot{}\rfoot{}
\parindent0pt

\begin{document}

%% layout for cover page
\thispagestyle{empty}
\parskip0pt

%% title box
\begin{center}\LARGE\bf\begin{tabular}{|c|}
  \hline \gamename\\ \gamedate\\ Rules and Scenario\\ \hline
\end{tabular}\end{center}

\vfill\vfill

%% player side of the GM/player contract
The following are the rules for {\em\gamename}, a real-time,
real-space roleplaying game sponsored by the \groupname.
You are responsible for knowing these rules.  Many of them are
nigh-impossible to enforce and rely upon the honor system.  Do not
cheat.  Do not abuse loopholes.  Play fair.  Be your own harshest
critic.

\vfill

%% GM side of the GM/player contract
The {\bf gamemasters} ({\bf GMs}) run the game.  If you have any
problems or questions concerning the game, contact a GM.  Rulings they
make are final.  They may violate the letter of the rules to preserve
the spirit.  The GMs promise to be as fair and reasonable as possible.
Neither they nor these rules are perfect.

\vfill

%% have fun
This game is intended to be fun.  Getting into character, roleplaying,
being dramatic, and playing competitively can all increase the fun of
the game.  Do not take the game too seriously.  Even if you are
losing, keep a good attitude.  When the game is over, the real winners
are the players with the best stories.

\vfill

%% be safe
This is only a game.  Everyone involved should act with courtesy,
sportsmanship, patience, and taste.  The GMs may expel anyone they
believe to be violating the spirit of the rules or the game.  Emotions
may run high.  If you think things are crossing the line from game to
reality too much, or if you are just getting too stressed, calm down
and maybe take a break.  Stay in control.  Use common sense.  Always,
play safely, then play to have fun.

\vfill

%% disclaimer and copyright
%% author list auto-generated from Lists/gm-LIST.tex
This game is a work of fiction.  Although it may refer to things in
the real world, it does so only for the sake of the scenario.  It does
not represent the opinions of the GMs or the  \groupname.
These rules are modifications of those used in previous games.  This
game and all materials thereof are copyright \the\year\ by
\SORT{GM}{alpha<}{\playeralphabetical}{\EVERY{GM}{}}{\ifcredit\MYplayer, \fi}%
and the \groupname.

\vfill\vfill

\begin{center}\bf
  Brought to you by the \groupname
\end{center}

\vfill

\clearpage

%% layout for Table of Contents page
%\thispagestyle{empty}
%\tableofcontents

%\clearpage

%% layout for main body of rules
\setcounter{page}{1}
\parskip5pt


%\section{Scenario}
%\clearpage


%% Some Assassin Game fundamentals
\section{Reality and Game Reality}

There is a big difference between reality and game reality.  Players
must treat each other with courtesy and explain to each other what
their characters perceive in confusing situations; e.g.\ ``My
character's hands are covered in blood,'' an {\bf out-of-game}
statement.  Characters are under no such restrictions, and may do what
it takes to further their goals; e.g.\ ``Uh, hi Bob.  Just got back
from the butcher shop,'' an {\bf in-game} statement.

{\bf Metagaming} is inferring in-game knowledge that is inappropriate
for your character from out-of-game information.  Do your best to not
metagame and especially to prevent the risk of metagaming.  Be your
own harshest critic.

\paragraph{Halts:} A halt pauses game action.  To call one, say ``game
halt'' in a clear and audible voice; other players around a corner
should hear you, but you shouldn't scare some poor grad student.  End
a halt by saying ``three, two, one, resume.''  Call a halt for one of
only three reasons: because a rule instructs you to, for safety and
similar out-of-game issues, or to pause game and fetch a GM (which you
should avoid).

\paragraph{Not-Here:} Putting a hand on your
head, visible from a distance, indicates that the player is not there.  Do not 
go not-here unless a rule instructs you to.

%% for shorter, intense games (SIK, etc.), add a NP Halt rule
\paragraph{Non-Players:} Use tact and common sense when dealing with
non-players ({\bf NPs}).   Avoid conspicuous or threatening game actions in front of NPs.
Shooting your friend outside of a classroom one minute before class
lets out is a bad idea, as is screaming bloody murder down a hallway.
If, despite your most valiant efforts, some NPs do get upset, call the
GMs who will help calm them down.

%% not for closed-space games
\paragraph{Observers:} An observer is someone not playing the game who
has agreed to watch.  They generally wear an observer headband or an
observer name-badge.  Observers have traditionally been called
``ghosts.''  They should stay out of the way; you can always ask an
observer to leave.  If a friend who is not playing wants to observe
game, send them to the GMs.

\section{Mechanics} Many actions your character can take, such as
walking, talking, and general interaction with other characters, are
represented by you doing them.  Others, like combat, are performed via
abstract mechanics, which are described in ability cards, greensheets,
and rules.  The abstract information for mechanics (like badge
numbers) may not be discussed in-game.  If you want to do something
special for which there is no mechanic, ask a GM.

Become familiar with your mechanics before game starts, especially
those which occur under time-pressure (like combat).  Game action will
not stop for memory packets, greensheets, or such.

A {\bf kludge} (and derivative forms like ``kludge-ite'') is something
impervious to logic and cleverness, usually for game-balance.  You
can't affect a kludge without a specified mechanic.

{\bf Zone of Control} ({\bf ZoC}) is a rough distance measurement.
You are within ZoC of someone if your outstretched fingers can touch
their outstretched fingers.  Double-ZoC is twice this distance,
triple-ZoC is three times, etc.

{\bf Headbands} represent obvious visual effects; wear them visibly on
your head.  If you see a headband and don't know what it represents,
ask.  If you are wearing a headband, tell people what their characters
see.

An {\bf interruptible} mechanic has some duration, and may involve
continuous roleplaying.  It is stopped if you are attacked or if
someone within ZoC says {\bf ``I stop you''} or an equivalent phrase.
Some mechanics may be easier or harder to interrupt.

A {\bf n-count} is an interruptible mechanic with a repeated, counted
incant (``I pour a drink one, I pour a drink two, I pour a drink
three'').  Speak clearly; each count must take at least a full second.
Each n-count will specify the number, e.g.\ a 3-count.

%To play {\bf Rock, Paper, Scissors} ({\bf RPS}), you and your
%opponent(s) say ``one, two, three, show'' in unison.  On ``show''
%everyone displays and compares their chosen symbol.  Rock is a closed
%fist.  Paper is a flat hand with palm down.  Scissors is a fist with
%the first two fingers extended, looking vaguely like a pair of
%scissors.  Rock defeats (crushes) scissors, scissors defeats (cuts)
%paper, paper defeats (covers) rock, and any symbol ties with itself.
%You may see or be able to play other, special symbols; the wielder
%will know what happens.

\paragraph{Safety:} This is a game.  Real violence is unacceptable.
Game action should cause no real-world damage, either to people or
property.  If something dangerous is happening, call a halt.  Stay in
control, use common sense, and do not endanger yourself or others.
You should not run or otherwise force your way into or through someone
else's ZoC, and you should not make physical contact with another
player without permission.


\section{Items Etc.}

Many in-game items are represented by little white cards with a number
and description.  Item cards may be shown to others, passed around,
stolen, etc.  The {\bf item number} on the card is not in-game
information and may not be discussed.  Not all in-game items have
cards or numbers; whatever they are represented by should be clearly
marked ``in-game item'' or ``freely transferable.''

\paragraph{Envelopes:} Some items and locations may have an attached
envelope (or just be a labeled packet or folded paper).  The envelope
may include directions for when to open these (``open packet if you
press the big red button'' or ``open packet if you eat this'');
otherwise you may only open them if instructed.  Close them when you
are done.  Open and close packets gently.

\paragraph{Signs:} Some locations and other game materials are
represented by signs or packets posted throughout game area.  You may
read any signs and must follow any rules printed on them.  If a sign
or packet doesn't have some sort of in-game description (it only has
out-of-game mechanics information, like a number or just a colored
dot), then your character doesn't even see it or know that anything
unusual is there.

\paragraph{Props:} Some items may have props (physical representations
or {\bf physreps}) associated with them.  The card and physrep should
be kept together.  If they are separated, the card is the real item.
Prop items are as bulky as the physrep.  They can be carried in bags
that can hold them, on straps that are attached to them, etc.

\section{Closing Notes}

These rules are imperfect.  The GMs may violate the letter of the
rules to preserve the spirit.  We hope these rules are reasonably
clear, but if you have any doubts about your interpretation, talk it
over with us in advance.  We should also add, as much as we hate to
admit it, we GMs are human: when all of our carefully laid plans are
going haywire, we may lose our cool.  The best way to deal with people
is remaining calm and friendly, especially when everyone is tired and
hungry.

We hope you have lots of fun.  Good luck.

\end{document}
